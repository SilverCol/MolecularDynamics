\documentclass[a4paper]{article}
\usepackage[slovene]{babel}
\usepackage[utf8]{inputenc}
\usepackage[T1]{fontenc}
%\usepackage[margin=2cm, bottom=3cm, foot=1.5cm]{geometry}
\usepackage{float}
\usepackage{graphicx}
\usepackage{amsmath}
\usepackage{subcaption}
\usepackage{hyperref}

\newcommand{\tht}{\theta}
\newcommand{\Tht}{\Theta}
\newcommand{\dlt}{\delta}
\newcommand{\eps}{\epsilon}
\newcommand{\thalf}{\frac{3}{2}}
\newcommand{\ddx}[1]{\frac{d^2#1}{dx^2}}
\newcommand{\ddr}[2]{\frac{\partial^2#1}{\partial#2^2}}
\newcommand{\mddr}[3]{\frac{\partial^2#1}{\partial#2\partial#3}}

\newcommand{\der}[2]{\frac{d#1}{d#2}}
\newcommand{\pder}[2]{\frac{\partial#1}{\partial#2}}
\newcommand{\half}{\frac{1}{2}}
\newcommand{\forth}{\frac{1}{4}}
\newcommand{\q}{\underline{q}}
\newcommand{\p}{\underline{p}}
\newcommand{\x}{\underline{x}}
\newcommand{\liu}{\hat{\mathcal{L}}}
\newcommand{\bigO}[1]{\mathcal{O}\left( #1 \right)}
\newcommand{\pauli}{\mathbf{\sigma}}
\newcommand{\bra}[1]{\langle#1|}
\newcommand{\ket}[1]{|#1\rangle}
\newcommand{\id}[1]{\mathbf{1}_{2^{#1}}}

\begin{document}

    \title{\sc\large Višje računske metode\\
		\bigskip
		\bf\Large Molekularna dinamika}
	\author{Mitja Vodnik, 28182041}
	\date{\today}
	\maketitle

    Obravnavamo Heisenbergovo spinsko verigo $N$ spinov $S = \half$, kjer je $N$ sodo naravno število.
    Uporabljamo izotropen antiferomagneten ($J = -1$) model, z interakcijo le med najbližjimi sosedi:

    \begin{equation}\label{eq1}
        H = \sum_{j=1}^{N} \pauli_j \cdot \pauli_{j+1}
    \end{equation}

    \iffalse
    \begin{figure}
        \centering
        \includegraphics[width = \textwidth]{slika1.pdf}
        \caption{Avtokorelacijska funkcija spinskega toka.}
        \label{slika1}
    \end{figure}
    \fi


\end{document}
