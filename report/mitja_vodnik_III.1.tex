\documentclass[a4paper]{article}
\usepackage[slovene]{babel}
\usepackage[utf8]{inputenc}
\usepackage[T1]{fontenc}
%\usepackage[margin=2cm, bottom=3cm, foot=1.5cm]{geometry}
\usepackage{float}
\usepackage{graphicx}
\usepackage{amsmath}
\usepackage{subcaption}
\usepackage{hyperref}

\newcommand{\tht}{\theta}
\newcommand{\Tht}{\Theta}
\newcommand{\dlt}{\delta}
\newcommand{\eps}{\epsilon}
\newcommand{\thalf}{\frac{3}{2}}
\newcommand{\ddx}[1]{\frac{d^2#1}{dx^2}}
\newcommand{\ddr}[2]{\frac{\partial^2#1}{\partial#2^2}}
\newcommand{\mddr}[3]{\frac{\partial^2#1}{\partial#2\partial#3}}

\newcommand{\der}[2]{\frac{d#1}{d#2}}
\newcommand{\pder}[2]{\frac{\partial#1}{\partial#2}}
\newcommand{\half}{\frac{1}{2}}
\newcommand{\forth}{\frac{1}{4}}
\newcommand{\q}{\underline{q}}
\newcommand{\p}{\underline{p}}
\newcommand{\x}{\underline{x}}
\newcommand{\liu}{\hat{\mathcal{L}}}
\newcommand{\bigO}[1]{\mathcal{O}\left( #1 \right)}
\newcommand{\pauli}{\mathbf{\sigma}}
\newcommand{\bra}[1]{\langle#1|}
\newcommand{\ket}[1]{|#1\rangle}
\newcommand{\id}[1]{\mathbf{1}_{2^{#1}}}
\newcommand{\tinv}{\frac{1}{\tau}}

\begin{document}

    \title{\sc\large Višje računske metode\\
		\bigskip
		\bf\Large Molekularna dinamika}
	\author{Mitja Vodnik, 28182041}
	\date{\today}
	\maketitle

    Obravnavamo problem prevajanja toplote v eni dimenziji.
    Imamo verigo $N$ atomov z enakimi masami ($m = 1$), z medatomskim potencialom $V(q_{j+1} - q_j)$ in s potencialom
    $U(q_j)$, ki vsak atom veže na periodičen substrat.
    Poleg tega, sta prvi in zadnji delec potopljena v različna termostata s temperaturama $T_L$ in $T_R$.
    Celotno Hamiltonsko funkcijo verige zapišemo kot:

    \begin{equation}\label{eq1}
        H = \half \sum_{j=1}^{N} p_j^2 + \sum_{j=1}^N U(q_j) + \sum_{j=1}^{N-1} V(q_{j+1} - q_j)
    \end{equation}

    Konkretno nas zanima transport toplote v verigi kvartičnih osclilatorjev, kjer potenciala zapišemo kot:

    \begin{equation}\label{eq2}
        V(x) = \half x^2, \quad U(x) = \half x^2 + \lambda x^4
    \end{equation}

    Termostata na konceh verige upoštevamo z dvema različnima modeloma toplotnih kopeli: Nose-Hooverjevim in Maxwellovim.

    \section{Nose-Hooverjev model}

    To je determinističen model toplotnih kopeli in sestoji iz naslednjega sistema enačb (za $j = 1, \ldots, N$):

    \begin{equation}\label{eq3}
        \begin{split}
            \der{q_j}{t} &= p_j\\
            \der{p_j}{t} &= -\pder{V(\q)}{q_j} - \delta_{j,1}\zeta_L p_1 - \delta_{j,N}\zeta_R p_N\\
            \der{\zeta_L}{t} &= \tinv \left( p_1^2 - T_L \right)\\
            \der{\zeta_R}{t} &= \tinv \left( p_N^2 - T_R \right)
        \end{split}
    \end{equation}

    Tu so $q_j$ in $p_j$ posplošene koordinate in impulzi delcev, $\zeta_L$ in $\zeta_R$ pa eksterni spremenljivki, ki
    delce vežeta s termostatoma. $V(\q)$ v tem sistemu prestavlja celoten potencial, torej:

    \begin{equation}\label{eq4}
        V(\q) = \half \sum_{j=1}^N q_j^2 + \lambda \sum_{j=1}^N q_j^4 + \half \sum_{j=1}^{N-1} (q_{j+1} - q_j)^2
    \end{equation}

    \iffalse
    \begin{figure}
        \centering
        \includegraphics[width = \textwidth]{slika1.pdf}
        \caption{Avtokorelacijska funkcija spinskega toka.}
        \label{slika1}
    \end{figure}
    \fi


\end{document}
