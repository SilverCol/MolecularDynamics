\documentclass[a4paper]{article}
\usepackage[slovene]{babel}
\usepackage[utf8]{inputenc}
\usepackage[T1]{fontenc}
%\usepackage[margin=2cm, bottom=3cm, foot=1.5cm]{geometry}
\usepackage{float}
\usepackage{graphicx}
\usepackage{amsmath}
\usepackage{subcaption}
\usepackage{hyperref}

\newcommand{\tht}{\theta}
\newcommand{\Tht}{\Theta}
\newcommand{\dlt}{\delta}
\newcommand{\eps}{\epsilon}
\newcommand{\thalf}{\frac{3}{2}}
\newcommand{\ddx}[1]{\frac{d^2#1}{dx^2}}
\newcommand{\ddr}[2]{\frac{\partial^2#1}{\partial#2^2}}
\newcommand{\mddr}[3]{\frac{\partial^2#1}{\partial#2\partial#3}}

\newcommand{\der}[2]{\frac{d#1}{d#2}}
\newcommand{\pder}[2]{\frac{\partial#1}{\partial#2}}
\newcommand{\half}{\frac{1}{2}}
\newcommand{\forth}{\frac{1}{4}}
\newcommand{\q}{\underline{q}}
\newcommand{\p}{\underline{p}}
\newcommand{\x}{\underline{x}}
\newcommand{\liu}{\hat{\mathcal{L}}}
\newcommand{\bigO}[1]{\mathcal{O}\left( #1 \right)}
\newcommand{\pauli}{\mathbf{\sigma}}
\newcommand{\bra}[1]{\langle#1|}
\newcommand{\ket}[1]{|#1\rangle}
\newcommand{\id}[1]{\mathbf{1}_{2^{#1}}}
\newcommand{\tinv}{\frac{1}{\tau}}

\begin{document}

    \title{\sc\large Višje računske metode\\
		\bigskip
		\bf\Large Molekularna dinamika}
	\author{Mitja Vodnik, 28182041}
	\date{\today}
	\maketitle

    Obravnavamo problem prevajanja toplote v eni dimenziji.
    Imamo verigo $N$ atomov z enakimi masami ($m = 1$), z medatomskim potencialom $V(q_{j+1} - q_j)$ in s potencialom
    $U(q_j)$, ki vsak atom veže na periodičen substrat.
    Poleg tega, sta prvi in zadnji delec potopljena v različna termostata s temperaturama $T_L$ na levi in $T_R$ na
    desni strani verige.
    Celotno Hamiltonsko funkcijo verige zapišemo kot:

    \begin{equation}\label{eq1}
        H = \half \sum_{j=1}^{N} p_j^2 + \sum_{j=1}^N U(q_j) + \sum_{j=1}^{N-1} V(q_{j+1} - q_j)
    \end{equation}

    Konkretno nas zanima transport toplote v verigi kvartičnih osclilatorjev, kjer potenciala zapišemo kot:

    \begin{equation}\label{eq2}
        V(x) = \half x^2, \quad U(x) = \half x^2 + \lambda x^4
    \end{equation}

    Termostata na konceh verige upoštevamo z dvema različnima modeloma toplotnih kopeli: Nose-Hooverjevim in Maxwellovim.
    Pri računanju bomo temperaturi termostatov vedno postavili na $T_L = 2$ in $T_R = 1$.

    \section{Nose-Hooverjev model}

    To je determinističen model toplotnih kopeli in sestoji iz naslednjega sistema enačb (za $j = 1, \ldots, N$):

    \begin{equation}\label{eq3}
        \begin{split}
            \der{q_j}{t} &= p_j\\
            \der{p_j}{t} &= -\pder{V(\q)}{q_j} - \delta_{j,1}\zeta_L p_1 - \delta_{j,N}\zeta_R p_N\\
            \der{\zeta_L}{t} &= \tinv \left( p_1^2 - T_L \right)\\
            \der{\zeta_R}{t} &= \tinv \left( p_N^2 - T_R \right)
        \end{split}
    \end{equation}

    Tu so $q_j$ in $p_j$ posplošene koordinate in impulzi delcev, $\zeta_L$ in $\zeta_R$ pa eksterni spremenljivki, ki
    delce vežeta s termostatoma. $V(\q)$ v tem sistemu predstavlja celoten potencial, torej:

    \begin{equation}\label{eq4}
        V(\q) = \half \sum_{j=1}^N q_j^2 + \lambda \sum_{j=1}^N q_j^4 + \half \sum_{j=1}^{N-1} (q_{j+1} - q_j)^2
    \end{equation}

    Sistem enačb v tem modelu ni Hamiltonski, zato si ne moremo pomagati s simplektičnimi integratorji - namesto tega
    uporabljamo metodo Runge-Kutta $4.$ reda.

    \section{Maxwellov model}

    Za razliko od prejšnjega, je ta model toplotnih kopeli stohastičen.
    Algoritem je naslednji:

    \begin{enumerate}
        \item Izberemo čas vzorčenja $\tau > 0$.
        \item Med časi $(n - 1)\tau$ in $n\tau$ dinamiko verige simuliramo s simplektičnim integratorjem za Hamiltonov
        operator \ref{eq1}.
        Ob časih $n\tau$ resetiramo momente $p_k$ delcev, ki so sklopljeni s kopelmi pri temperaturah $T_k$, tako da
        jih izžrebamo po porazdelitvi:
        \begin{equation}\label{eq5}
            W(p_k) = \frac{1}{\sqrt{2\pi}\sigma_k} exp\left( -\frac{p_k^2}{2\sigma_k^2} \right),
            \quad \sigma_k = \sqrt{T_k}
        \end{equation}
    \end{enumerate}

    Parameter $\tau$ mora  biti  izbran  primerno,  da  bodo  kontakti  s  koplemi  kar najmočnejši, sicer dobimo t.i.
    kontaktne upornosti.

    \section{Temperaturni profil}

    Oglejmo si profil efektivne temperature:

    \begin{equation}\label{eq6}
        T_j = \langle p_j^2 \rangle = \frac{1}{t} \int_0^t p_j(t') dt'
    \end{equation}

    Računamo ga v dveh režimih: harmonskem ($\lambda = 0$) in anharmonskem ($\lambda = 1$).
    V harmonskem režimu pričakujemo konstanten temperaturni profil: $\langle p_j^2 \rangle = \frac{T_L + T_R}{2}$, v
    anharmonskem pa linearnega: $\langle p_j^2 \rangle \simeq T_L + \frac{j - 1}{N - 1}(T_R - T_L)$. \\

    Na slikah \ref{slika1} in \ref{slika3} sta temperaturna profila harmonske verige izračunana z različnima modeloma
    termostatov.
    Oba se približno držita pričakovane konstante.
    Na slikah \ref{slika2} in \ref{slika4} pa se približno kažeta pričakovana linearna temperaturna profila.


    \begin{figure}
        \centering
        \begin{subfigure}{\textwidth}
            \includegraphics[width = \textwidth]{slika3.pdf}
            \caption{Harmonska veriga: $\lambda = 0$.
            Oranžna črta označuje povprečno vrednost vmesnih vrednosti.}
            \label{slika1}
        \end{subfigure}
        \begin{subfigure}{\textwidth}
            \includegraphics[width = \textwidth]{slika10.pdf}
            \caption{Anharmonska veriga: $\lambda = 1$}
            \label{slika2}
        \end{subfigure}
        \caption{Temperaturna profila za harmonsko in anharmonsko verigo dolžine $N = 40$ povprečena po času $t = 10^6$.
        Uporabljen je bil Nose-Hooverjev termostat.}
    \end{figure}

    \begin{figure}
        \centering
        \begin{subfigure}{\textwidth}
            \includegraphics[width = \textwidth]{slika7.pdf}
            \caption{Harmonska veriga: $\lambda = 0$.
            Oranžna črta označuje povprečno vrednost vmesnih vrednosti.}
            \label{slika3}
        \end{subfigure}
        \begin{subfigure}{\textwidth}
            \includegraphics[width = \textwidth]{slika14.pdf}
            \caption{Anharmonska veriga: $\lambda = 1$}
            \label{slika4}
        \end{subfigure}
        \caption{Temperaturna profila za harmonsko in anharmonsko verigo dolžine $N = 80$ povprečena po času $t = 10^6$.
        Uporabljen je bil Maxwellov termostat.}
    \end{figure}

    \section{Energijski tok}

    Energijski tok na mestu $j$ znotraj verige dobimo kot:

    \begin{equation}\label{eq7}
        \langle J_j \rangle = -\half \left( V'(q_j - q_{j - 1}) + V'(q_{j + 1} - q_j) \right)p_j
        = \half (q_{j - 1} - q_{j + 1})p_j
    \end{equation}

    Za anharmonsko verigo pričakujemo transport energije, ki ga narekuje Fourijerjev zakon:

    \begin{equation}\label{eq8}
        \langle J_j \rangle \simeq \kappa \frac{T_R - T_L}{N},
    \end{equation}

    kjer je $\kappa$ konstanta toplotne prevodnosti.
    Približno ustrezanje temu zakonu je prikazano na sliki \ref{slika5}.
    Vidimo, da se konstanti za različna modela termostatov razlikujeta za približno faktor $2$.

    \begin{figure}
        \centering
        \includegraphics[width = \textwidth]{slika15.pdf}
        \caption{Energijski tokovi in konstanti toplotne prevodnosti za različna tipa termostatov.
        Tokovi so računani pri dolžinah $N = 10, 20, 40$ in anharmonski konstanti $\lambda = 1$, ter povprečeni po času
        $t = 10^6$.
        Na rezultate je prilagojena premica $\langle p_j \rangle = \frac{\kappa}{N}$.}
        \label{slika5}
    \end{figure}

    Izračunajmo še toplotna tokova med verigo in termostatoma:

    \begin{equation}\label{eq9}
        J = \lim_{N \to \infty} \frac{1}{N\tau} \sum_{n = 1}^N \half \{ p_k^2(t=n\tau + 0) - p_k^2(t=n\tau - 0) \},
    \end{equation}

    kjer je $p_k$ impulz delca sklopljenega s termostatom, zapis $t = n\tau \pm 0$ pa pomeni, da izvrednotimo tik pred
    oziroma po žrebanju nove vrednosti momenta v Maxwellovem algoritmu.
    Odvisnost tako dobljenega toka med termostatoma in verigo od vrednosti vzorčnega časa $\tau$ (slika \ref{slika6}) nam
    pomaga pri pravilni izbiri vzorčnega časa - želimo si čim manjše kontaktne upornosti oz. čim večjega toka.

    \begin{figure}
        \centering
        \includegraphics[width = \textwidth]{slika16.pdf}
        \caption{Odvisnost toka $J$ med termostatom in verigo v odvisnosti od časa vzorčenja $\tau$.
        Računano je na anharmonski ($\lambda = 1$) verigi dolžine $N = 10$ in povprečeno po $1000$ vzorcih.
        $J_1$ se nanaša na tok na levi, $J_2$ pa na desni strani verige.}
        \label{slika6}
    \end{figure}

\end{document}
